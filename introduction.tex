%!TEX root = these.tex

%\chapterstar{INTRODUCTION}
% Pour des renseignements sur \chapterstar : voir le fichier macros.tex

\chapter*{Introduction} % "*" pour que l'introduction ne s'affiche pas dans la table des matières, sinon elle s'y affichera comme un chapitre d
% \mtcaddchapter
% \addstarredpart{Introduction} % Pour ajouter une partie ("part") fictive dans la table des matières
% \mtcaddpart
% \markboth{Introduction}{Introduction}  %% header manuel car sinon, ils me marquent le header du dernier \chapter{?} (on est dans un \chapter*{?} )
\selectlanguage{francais}


\section*{Contexte et problématique}
	La sélection de cibles en mouvement est une tâche récurrente dans de nombreuses applications allant des jeux vidéo aux simulations moléculaires interactives, en passant par les interfaces dédiées au contrôle aérien, par les vidéos interactives, comme les enregistrements d'événements sportifs ou la vidéo-surveillance, par la mécanique des fluides et les applications de défense et sécurité. Dans ce dernier domaine, la tâche de sélection peut être critique, compte tenu des enjeux.
	
	Si la sélection de cibles immobiles a fait l'objet de nombreuses études, la sélection de cibles en mouvement a été assez peu abordée dans la littérature scientifique, car les facteurs qui caractérisent ce mouvement peuvent être nombreux, variés et complexes : la rapidité des cibles, leur densité, leur occultation visuelle, l'imprévisibilité de leurs mouvements\ldots{} En effet, alors qu’avec des cibles immobiles, certains modèles tels que la loi de Fitts permettent d’estimer la difficulté de sélection, ils en sont incapables pour des cibles mobiles.
	
	Certains modèles caractérisant la sélection ou le pointage de cibles mobiles existent, mais ils sont généralement limités aux cibles de mouvements uniformes, c'est-à-dire rectilignes et de vitesse constante. Bien que potentiellement précis, ces modèles ne sauraient satisfaire l'ensemble des besoins liés à la sélection de cibles mobiles, et l'influence de facteurs décrivant le comportement dynamique d'une cible sur les performances de sélection reste à déterminer. De surcroît, la sélection de cibles mobiles est (souvent beaucoup) plus difficile qu'avec des cibles statiques, et il est d'autant plus important de mieux la comprendre.
	
	De même, les techniques d'assistance à la sélection spécifiquement dédiées aux cibles mobiles sont peu nombreuses, ce qui est peut-être dû à la pauvreté de la littérature théorique, c'est-à-dire au manque de modèles. De telles techniques existent, mais sont généralement fondées sur les principes de la sélection statique, éventuellement adaptés à un contexte dynamique, plus que sur des recommandations spécifiques aux cibles mobiles. D'autres techniques tentent de prédire l'intention de l'utilisateur, toujours sans fondement théorique --- ce qui n'exclut pas une certaine efficacité.
	
	Mais souvent, les techniques existantes présentent des inconvénients significatifs, surtout pour les contextes les plus difficiles, à savoir ceux qui présentent les cibles les plus rapides, imprévisibles, nombreuses et occultées.

\section*{Plan du manuscrit}
	Dans le premier chapitre de ce manuscrit, nous dresserons un inventaire des applications nécessitant une tâche de sélection de cible mobile. Dans chaque cas, nous détaillerons le contexte, préciserons la nature des besoins, des enjeux et des difficultés. Nous nous attarderons volontairement sur les simulations de dynamique moléculaire, qui constituent la motivation originelle de ces travaux, mais n'oublierons pas le contrôle des espaces terrestre, maritime, aérien et extra-atmosphérique, le sport, la surveillance des espaces publics ou les applications vidéoludiques.
	
	Dans le deuxième chapitre, nous ferons, après quelques rappels sur la théorie de base du pointage, un bref état de l'art des modèles de pointage et de sélection de cibles, notamment mobiles. Puis nous établirons un état de l'art des techniques d'assistance à la sélection de cibles, que nous classifierons selon leurs caractéristiques respectives, et notamment en fonction de leur pertinence pour la sélection de cibles mobiles.
	
	Le troisième chapitre vise à créer une taxinomie des environnements de sélection, c'est-à-dire des cibles elles-mêmes en fonction de leurs caractéristiques, mais aussi des environnements dans lesquels elles évoluent. Nous ferons une liste des critères nous permettant d'établir cette taxinomie, en les justifiant, puis nous les appliquerons aux cibles et environnements détaillés dans le premier chapitre. Nous détaillerons ensuite un modèle que nous avons développé pour décrire et surtout générer du mouvement aléatoire, puis nous caractériserons certains environnements de sélection en fonction des critères de ce modèle.
	
	Dans le quatrième et dernier chapitre, nous reviendrons sur ce modèle, ses limites et possibilités d'extension, avant de décrire en détail l'évaluation empirique que nous avons menée pour caractériser les performances de sélection de cibles mobiles en fonction de la nature de leurs mouvements. Nous examinerons ainsi l'impact des différents paramètres de notre modèle de génération de mouvement sur les performances, d'abord séparément, puis ensemble. Nous nous appuierons sur ces mesures quantitatives et sur les impressions subjectives de nos sujets pour mieux comprendre l'origine des difficultés lors d'une tâche de sélection. Nous nous attarderons sur les différents profils de vitesse du curseur selon la nature des mouvements des cibles, afin d'estimer l'importance des phases dites balistique et de correction au cours de la tâche. Puis, nous proposerons une approche de prédiction des performances de sélection fondée sur l'aire de l'enveloppe convexe de la trajectoire d'une cible mobile, à l'aide de différents modèles.