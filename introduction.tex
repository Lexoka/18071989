%!TEX root = these.tex

%\chapterstar{INTRODUCTION}
% Pour des renseignements sur \chapterstar : voir le fichier macros.tex

\chapter*{Introduction} % "*" pour que l'introduction ne s'affiche pas dans la table des matières, sinon elle s'y affichera comme un chapitre d
% \mtcaddchapter
% \addstarredpart{Introduction} % Pour ajouter une partie ("part") fictive dans la table des matières
% \mtcaddpart
% \markboth{Introduction}{Introduction}  %% header manuel car sinon, ils me marquent le header du dernier \chapter{?} (on est dans un \chapter*{?} )
\selectlanguage{francais}


\section*{Contexte et problématique}
    La sélection de cibles mobiles est une tâche que l'on retrouve dans de nombreux domaines. Le contrôle du trafic aérien est une application critique, mais relativement aisée du fait, d'une part aux jeux vidéo, en passant par les vidéos interactives (enregistrements d'événements sportifs ou vidéo-surveillance), les simulations scientifiques (mécanique des fluides ou dynamique moléculaire), et les applications militaires. Si la sélection de cibles statiques est un problème bien connu, modélisé par la loi de Fitts et ses nombreuses extensions, et facilité par de nombreuses techniques d'assistance, la sélection de cibles mobiles est plus difficile.
    



 
\section*{Approche générale}


\section*{Plan du manuscrit}
