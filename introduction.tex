%!TEX root = these.tex

%\chapterstar{INTRODUCTION}
% Pour des renseignements sur \chapterstar : voir le fichier macros.tex

\chapter*{Introduction} % "*" pour que l'introduction ne s'affiche pas dans la table des matières, sinon elle s'y affichera comme un chapitre d
% \mtcaddchapter
% \addstarredpart{Introduction} % Pour ajouter une partie ("part") fictive dans la table des matières
% \mtcaddpart
% \markboth{Introduction}{Introduction}  %% header manuel car sinon, ils me marquent le header du dernier \chapter{?} (on est dans un \chapter*{?} )
\selectlanguage{francais}


\section*{Contexte et problématique}
	La sélection de cibles en mouvement est une tâche récurrente dans de nombreuses applications allant des jeux vidéo aux simulations moléculaires interactives, en passant par les interfaces dédiées au contrôle aérien, par les vidéos interactives, comme les enregistrements d'événements sportifs ou la vidéo-surveillance, par la mécanique des fluides et les applications de défense et sécurité. Dans ce dernier domaine, la tâche de sélection peut être critique, compte tenu des enjeux.
	
	Si la sélection de cibles immobiles a fait l'objet de nombreuses études, la sélection de cibles en mouvement a été assez peu abordée dans la littérature scientifique, car les facteurs qui caractérisent ce mouvement peuvent être nombreux, variés et complexes : la rapidité des cibles, leur densité, leur occultation visuelle, l'imprévisibilité de leurs mouvements\ldots{} En effet, alors qu’avec des cibles immobiles, certains modèles tels que la loi de Fitts permettent d’estimer la difficulté de sélection, ils en sont incapables pour des cibles mobiles.
	
	Certains modèles caractérisant la sélection ou le pointage de cibles mobiles existent, mais ils sont généralement limités aux cibles de mouvements uniformes, c'est-à-dire rectilignes et de vitesse constante. Bien que potentiellement précis, ces modèles ne sauraient satisfaire l'ensemble des besoins liés à la sélection de cibles mobiles, et l'influence de facteurs décrivant le comportement dynamique d'une cible sur les performances de sélection reste à déterminer. De surcroît, la sélection de cibles mobiles est (souvent beaucoup) plus difficile qu'avec des cibles statiques, et il est d'autant plus important de mieux la comprendre.

\section*{Plan du manuscrit}
	Dans le premier chapitre de ce manuscrit, nous dresserons un inventaire des applications nécessitant une tâche de sélection de cible mobile. Dans chaque cas, nous détaillerons le contexte, préciserons la nature des besoins, des enjeux et des difficultés. Nous nous attarderons volontairement sur les simulations de dynamique moléculaire, qui constituent la motivation originelle de ces travaux, mais n'oublierons pas le contrôle des espaces terrestre, maritime, aérien et extra-atmosphérique, le sport, la surveillance des espaces publics ou les applications vidéoludiques.
	
	Dans le deuxième chapitre, nous ferons un état de l'art des techniques d'assistance à la sélection de cibles,