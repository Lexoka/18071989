%!TEX root = these.tex

\chapter*{Notations et expressions}

\begin{table}[htbp]
\centering
\begin{tabular}{l l l}

\textbf{Acronyme ou notation} & \textbf{Signification / Traduction} & \\ \\ %\textbf{Page de définition} \\ \\
\hline
\\

$IHM$ & Interaction Homme-Machine & \\

\end{tabular}
\end{table}

\begin{description}
    \item[Alcoxyle ou alkoxy.] En chimie, un groupe alcoxyle (souvent désigné par groupe alkoxy) est une espèce chimique de type monoradicalaire constituée d'un groupe alkyle lié à un atome d'oxygène.
    \item[Fonction ester.] En chimie, la fonction ester désigne un groupement caractéristique formé d'un atome lié simultanément à un atome d'oxygène par une double liaison et à un groupement alkoxy du type R-COO-R'.
    \item[Liaison phosphodiester.] La liaison phosphodiester correspond au lien entre le phosphore d'un groupement phosphate avec deux autres molécules via 2 liens ester, il s'agit donc en fait de deux liaisons phosphoester. Dans l’ADN et l’ARN, la liaison phosphodiester correspond au lien entre deux nucléotides par leurs carbones 3’ et 5’ du désoxyribose ou du ribose.
\end{description}



% \begin{acronym}
% 	\acro{ADN}{Acide Desoxyribo-Nucléique}
% 	\acro{ARN}{Acide Ribo-Nucléique}
% \end{acronym}