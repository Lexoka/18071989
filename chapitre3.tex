%!TEX root = these.tex

\chapter[Chapitre3]{Chapitre3}
\minitoc
\label{chap3}
\cleardoublepage

\section{Introduction}
	Au cours du premier chapitre de ce manuscrit, nous avons identifié un certain nombre de besoins associés à des applications précises. Nous nous sommes également attachés à caractérisés la nature des cibles mobiles que l'on rencontre dans ces applications, afin de permettre au lecteur d'apprécier d'une part l'ensemble des difficultés inhérentes aux tâches de sélection dans ces applications, et d'autre part la nécessité d'une assistance à la sélection.
	
	Si nous avons jusqu'ici énuméré et caractérisé ces types de cibles, avec une attention toute particulière à leur contexte applicatif, nous n'avons pas procédé à une classification systématique des cibles selon la nature de leur mouvement, définie selon des critères objectifs et et des mesures quantitatives.
	
	Or, il nous apparaît que pour réellement comprendre les enjeux et défis liés à la sélection de cibles mobiles, une telle classification est nécessaire. L'objectif de ce chapitre est donc d'établir une taxinomie des cibles mobiles selon des critères objectifs et, dans la mesure du possible, permettant une quantification des valeurs auxquelles ils se rapportent.

\section{Modèle de génération de mouvement}

\section{Taxinomie des cibles mobiles}
\subsubsection{Mouvement autocorrélé}
\subsubsection{Mouvement non autocorrélé, ou aléatoire}

\section{Conclusion}


\clearpage
