%!TEX root = these.tex

\chapter[Taxinomie des environnements de sélection]{Taxinomie des environnements de sélection}
\minitoc
\label{chap3}
\cleardoublepage

\section{Introduction}

	\subsection{Nature des cibles}
	Au cours du premier chapitre de ce manuscrit, nous avons identifié un certain nombre de besoins associés à des applications précises. Nous nous sommes également attachés à caractériser la nature des cibles mobiles que l'on rencontre dans ces applications, afin de permettre au lecteur d'apprécier d'une part l'ensemble des difficultés inhérentes aux tâches de sélection dans ces applications, et d'autre part la nécessité d'une assistance à la sélection.
	
	Si nous avons jusqu'ici énuméré et caractérisé ces types de cibles, avec une attention toute particulière à leur contexte applicatif, nous n'avons pas procédé à une classification systématique des cibles selon la nature de leur mouvement, définie selon des critères objectifs et des mesures quantitatives.
	
	Or, il nous apparaît que pour réellement comprendre les enjeux et défis liés à la sélection de cibles mobiles, une telle classification est nécessaire. L'objectif de ce chapitre est donc d'établir une taxinomie des cibles mobiles selon des critères objectifs et, dans la mesure du possible, permettant une quantification des valeurs auxquelles ils se rapportent.
	
	\subsection{Nature des environnements}
	Bien qu'une \og simple \fg{} taxinomie des cibles mobiles en fonction de la nature de leur mouvement ait beaucoup d'intérêt, elle ne saurait fournir suffisamment d'informations pour guider la conception de techniques de sélection sans tenir compte de l'\emph{environnement} de sélection. En effet, la cible la plus petite, la plus rapide et la plus imprévisible qui soit est triviale à sélectionner s'il s'agit du seul objet d'intérêt dans l'environnement : il n'y a qu'une sélection possible, donc la technique de sélection optimale --- ou du moins suffisante --- consiste à permettre la sélection de la cible par la simple pression d'un bouton, ou activation d'un quelconque périphérique de saisie.
	
	En effet, du point de vue de la théorie de l'information de Shannon~\cite{shannon2001mathematical}, un seul bit d'information est à transmettre de l'utilisateur au système, correspondant à la réponse à la question suivante : \og la cible doit-elle être sélectionnée ? \fg{}. Si la réponse est négative, l'utilisateur ne fait rien et le système non plus ; si elle est positive, une seule action est nécessaire de la part de l'utilisateur, et le système, qui connaît la position de la cible, n'a qu'à la sélectionner sans requérir de précision de la part de l'utilisateur.
	
	Même dans un cas où il y aurait plusieurs objets de ce type, mais en petit nombre, la sélection demeurerait relativement aisée avec une technique telle que le \emph{Bubble Cursor}, analysé au cours du deuxième chapitre. En effet, cette technique illustrée par la figure~\ref{fig:bubble} partage l'espace en plusieurs cellules, selon un diagramme de Voronoï (voire figure~\ref{fig:voronoi}). De fait, avec par exemple quatre cibles (éventuellement très petites, rapides et imprévisibles) l'espace virtuel serait partagé en quatre parties qui, la plupart du temps, serait très grandes. La loi de Fitts ne s'appliquerait pas directement, car ces zones de sélection seraient mobiles, mais l'on voit bien que la sélection ne serait pas très difficile.
	
	À l'inverse, avec des cibles aussi petites, rapides et imprévisibles, mais extrêmement nombreuses, l'on comprend aisément que l'intérêt du \emph{Bubble Cursor} serait très fortement diminué car les cellules de Voronoï de chaque cible deviendraient fort petites, et pas nécessairement significativement plus grandes que les cibles elles-mêmes.
	
	Il apparaît donc clairement que la difficulté d'une tâche de sélection ne peut être évaluée qu'en tenant compte de l'environnement dans lequel l'objet ciblé est sélectionné. De fait, les besoins et contraintes devant orienter la conception d'une technique d'assistance doivent également en tenir compte. Aussi notre taxinomie tiendra-t-elle compte de l'environnement global, et non seulement de la cible à sélectionner et de la nature de ses mouvements.
	
	\subsection{Critères subjectifs}
	Nous verrons plus loin que des critères plus subjectifs --- quoique fondés sur des observations empiriques --- peuvent être plus pertinents pour estimer la difficulté d'une tâche de sélection, et nous proposerons une taxinomie révisée en conséquence.

\section{Taxinomie des environnements de sélection}

    \subsection{Critères de discrimination}
    L'établissement de notre taxinomie passe par le choix des critères qui nous permettront de d'établir des distinctions entre les types de mouvements des cibles et les environnements de sélection. Dans les sous-sections suivantes, nous allons détailler les critères que nous avons retenus.

    \subsubsection{Autocorrélation}
    Nous considérerons ici qu'un mouvement est autocorrélé si un \emph{changement de direction} opéré à l'instant $t$ dépend du \emph{changement} (éventuellement nul) opéré à l'instant $t-1$. Si, au contraire, un changement peut avoir lieu à l'instant $t$ quel que fût la situation à l'instant précédent, nous appellerons ce mouvement \emph{markovien}~\cite{markov1960theory} en admettant qu'il s'agit d'un abus de langage, puisqu'un processus respectant la propriété de Markov est totalement indépendant de l'état du système à l'instant précédent ; or, ici, le vecteur direction d'un objet à l'instant $t$ peut dépendre de son orientation à l'instant $t-t$ même si le changement d'orientation n'en dépend pas.
    
    En effet, si ledit changement se fait selon un angle borné (entre -45\textdegree{} et +45\textdegree{}, par exemple) alors l'orientation du vecteur direction à l'instant $t$ dépendra de son orientation à $t-1$. Tant que le changement de direction à l'instant $t$, lui, est bien indépendant du changement à l'instant $t-1$ nous admettrons cet abus de langage et parlerons de mouvement markovien ; sinon, le mouvement sera dit autocorrélé. Par commodité, un objet dont les mouvements sont autocorrélés sera dit autocorrélé, et un objet dont les mouvements sont markoviens sera dit markovien.
    
    Concrètement, les objets macroscopiques dont les mouvements sont soumis aux lois de la physique newtonienne~\cite{newton1833philosophiae} ont des mouvements autocorrélés --- ce sera donc le cas de véhicules de tous types, des êtres vivants (athlètes compris) des projectiles, missiles, etc. Les objets nanoscopiques dont les mouvements sont soumis aux lois de la mécanique quantique ont des mouvements markoviens --- ce sera notamment le cas des particules dans les simulations moléculaires. Les objets pouvant être sélectionnés dans les jeux vidéo seront autocorrélés ou markoviens, selon les règles choisies par les développeurs.

    \paragraph{Uniformité.}
    Le mouvement autocorrélé le plus simple est le mouvement uniforme, c'est-à-dire celui pour lequel le vecteur direction ne change jamais. Les objets de mouvement uniforme se déplacent donc en ligne droite. Inversement, un mouvement n'est pas uniforme si, à un instant quelconque, le vecteur direction de l'objet concerné change.

    \paragraph{Périodicité.}
    Un mouvement sera dit périodique si les changements de directions sont tels que l'objet effectuera une trajectoire fermée qu'il répétera à intervalles réguliers. Plus formellement, un mouvement est périodique s'il admet une période $T$ telle que :
    $\forall t,~Position_{t}~=~Position_{t+T}$ où $Position_{t}$ désigne la position de l'objet à l'instant $t$.
    
    \subparagraph{Pseudo-périodicité.}
    Nous appellerons pseudo-périodique un mouvement caractérisé par une trajectoire fermée et répétée à intervalles \emph{irréguliers}. Un tel mouvement ne satisfait pas la condition formalisée ci-dessus, mais admet un ensemble de positions limitées à une trajectoire donnée, et revisitées continuellement dans le même ordre --- simplement, à des vitesses pouvant varier.

    \subparagraph{Circularité.}
    Le mouvement circulaire est un cas particulier du mouvement périodique. Comme son nom l'indique assez clairement, il s'agit d'un mouvement suivant une trajectoire circulaire. Il peut également être pseudo-périodique.

    \subsubsection{Densité et occultation}
    Nombre de cibles par unité de surface/volume, occultation.

    \subsection{Taxinomie obtenue}
    Résultats ; bidule OpenCV ?

\section{Modèle de génération de mouvement}
    \subsection{Motivations et principe}
    Vitesse, angle, fréquence ; pas d'autocorrélation.

    \subsection{Résultats}
    Dynamique moléculaire. Foules ? Jeux vidéo ? Autres ?


\section{Conclusion}

\clearpage
