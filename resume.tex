%!TEX root = these.tex

%\skiptoevenpage 
%\addcontentsline{toc}{chapter}{Résumé / Abstract}
% \backmatter 
\chapterstar{Résumé / Abstract}% si on enlève le \chapter , le lien dans le pdf revoit vers la biblio 
\pagestyle{plain}

\setlength{\headheight}{0.pt}
% \thispagestyle{empty} 
%\pdfbookmark[0]{Résumé}{resume}   
%~ % pour que la page soit la bonne dans la table des matière 
        
% \includegraphics[height=1.cm]{./figures/LogoUPSUD.pdf}\hfill
% \includegraphics[height=1.cm]{./figures/limsilogo_vectoriel.pdf}\hfill


% \begin{center}
%  \textbf{Alexandre Kouyoudjian} \\
%  \textbf{blabla}
% \end{center}
    
%\fontsize{10}{12}
%\selectfont
\subsection*{Résumé}

%\tiny
%(minuscule)

%\scriptsize
%(très petit)

%\footnotesize
%(assez petit)

%\small
%(petit)

%\normalsize
% normal

%\large
%(grand)

%\Large
%(plus grand)

%\LARGE
%(très grand)

%\huge
%(énorme)

%\normalsize

\begin{otherlanguage}{english}
%  \vspace{1cm}
%
%  \begin{center} \rule{\textwidth/3}{1pt} \end{center}
%  \vspace{1cm}
\subsection*{Abstract}

La sélection de cibles en mouvement est une tâche récurrente dans de nombreuses applications allant des jeux vidéos aux simulations moléculaires interactives, en passant par les interfaces dédiées au contrôle aérien. Si la sélection de cible immobile a fait l’objet de nombreuses études, le sélection de cibles en mouvement a été assez peu abordée dans la littérature scientifique, car les facteurs qui caractérisent ce mouvement peuvent être nombreux, variés et complexes : la rapidité des cibles, la densité des cibles, leur occultation visuelle, l’imprévisibilité de leur mouvement... En effet, contrairement aux cibles immobiles, contexte dans lequel des modèles tel que la loi de Fitt permettent d’approximer la difficulté de sélection, l’influence de facteurs décrivant le comportement dynamique d’une cible sur les performances de sélection reste à déterminer.

Ces travaux de thèse ont d’abord consisté à formaliser le mouvement de cible mobile, à partir de paramètres du mouvement caractérisant l’aspect à la fois dynamique et imprévisible d’une cible mobiles. Dans ce modèle, un nombre de facteurs limités et faciles à contrôler dans le cadre d’une simulation de cibles en mouvements, ont été définis pour être compatibles avec une étude expérimentale : la vitesse (V), la période entre chaque changement de direction et la fréquence (F) correspondante, ainsi que l’amplitude angulaire maximale (A) de ces changements de direction entre deux instants. Ces paramètres ont aussi été choisi pour modéliser le comportement de cibles au mouvement erratique, comme les atomes dans le cadre de simulation interactive, application qui est la source de la problématique abordée dans cette thèse.

A partir de cette formalisation, une expérimentation a été menée pour connaître l’influence de ces paramètres sur les performances de sélection. Les résultats quantitatifs de cette expérience ont montré qu’il était difficile d’établir un modèle prédicatif permettant de déterminer un indice de difficulté de la tache de sélection, du fait d’une forte interaction entre ces paramètres. Des résultats qualitatifs nous ont indiqué par ailleurs que ce sont les combinaisons particulières de facteurs qui évoque chez les sujets différentes classes de mouvements, perçus comme « vibratoire » ou « brownien », « régulier » ou « imprévisible », directement reliés à la difficulté de sélection, aboutissant donc à différentes stratégies d’anticipation en fonction de la nature perçue du mouvement.

Ces résultats ont conduit à rechercher d’autres critères, notamment des descripteurs de trajectoires des cibles, qui dépendent des facteurs précédemment décrits, comme l’aire de la coque convexe de la trajectoire que décrit la cible sur une période, critères permettant de mieux prédire les performances de sélection d’une cible mobile.

Les résultats obtenus sont un pré-requis à la proposition de recommandations ergonomiques pour la conception de paradigmes d’interaction, incluant l’utilisation d’une heuristique de prédiction du mouvement de cible, qui couplée à une assistance haptique pseudo-haptique, aurait pour objectif de faciliter cette tache de sélection de cible mobile particulièrement difficile, en permettant notamment de savoir comment adapter les aides à l’interaction en fonction des facteurs qui caractérisent la dynamique et le degré de prédictibilité d’une cible. 


\textbf{Mots-clefs} : Piçking de cibles, cibles mobiles, indice de difficulté, aide à la sélection

\end{otherlanguage} 