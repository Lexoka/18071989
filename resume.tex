%!TEX root = these.tex

\chapter*{Résumé / Abstract}% si on enlève le \chapter , le lien dans le pdf revoit vers la biblio 
%\pagestyle{plain}

% \thispagestyle{empty} 
\addcontentsline{toc}{chapter}{Résumé / Abstract}
\mtcaddchapter % pour éviter décalage de minitoc
%~ % pour que la page soit la bonne dans la table des matière 

%\begin{otherlanguage}{english}
%  \vspace{1cm}
%
%  \begin{center} \rule{\textwidth/3}{1pt} \end{center}
%  \vspace{1cm}
%\subsection*{Abstract}
%\end{otherlanguage}

\pagestyle{plain}

%\subsection*{Résumé}
	La sélection de cibles en mouvement est une tâche récurrente dans de nombreuses applications allant des jeux vidéo aux simulations moléculaires interactives, en passant par les interfaces dédiées au contrôle aérien. Si la sélection de cibles immobiles a fait l'objet de nombreuses études, la sélection de cibles en mouvement a été assez peu abordée dans la littérature scientifique, car les facteurs qui caractérisent ce mouvement peuvent être nombreux, variés et complexes : la rapidité des cibles, leur densité, leur occultation visuelle, l'imprévisibilité de leurs mouvements\ldots{} En effet, alors qu’avec des cibles statiques, certains modèles tels que la loi de Fitts permettent d’estimer le temps de sélection, ils en sont incapables pour des cibles mobiles ; l'influence de facteurs décrivant le comportement dynamique d'une cible sur les performances de sélection reste à déterminer.

	Ces travaux de thèse ont d'abord consisté à dresser un inventaire des applications impliquant une tâche de sélection de cible mobile, en décrivant leurs caractéristiques et leurs enjeux, en tenant tout particulièrement compte de leur contexte.
	
	Ensuite, nous avons proposé un état de l'art des techniques de sélection de cibles, dont nous avançons une proposition de taxinomie, selon leur focus (cibles statiques ou mobiles), leur approche du problème, et leurs propriétés. Puis, nous avons formalisé une série de critères permettant de classifier les cibles mobiles en fonction de la nature de leurs mouvements d'une part, et des caractéristiques des environnements dans lesquels on les rencontre d'autre part. Nous avons appliqué ces critères aux différents types de cibles précédemment identifiés, afin d'en proposer une classification. En nous appuyant sur les critères principaux, nous avons proposé un modèle de description et de génération de mouvement, permettant à la fois de caractériser une cible mobile, et d'en animer une directement, de façon finement contrôlable. Cela nous a permis, via un travail d'annotation manuel de nombreuses vidéos de cibles mobiles diverses, d'en extraire les paramètres essentiels, en vue d'estimer la difficulté des tâches de sélection correspondantes. Dans ce modèle, appelé VFA, trois facteurs aisément manipulables ont été définis pour être compatibles avec une étude expérimentale : la vitesse (V), la période entre chaque changement de direction et la fréquence (F) correspondante, ainsi que l'amplitude angulaire maximale (A) de ces changements de direction. Ces paramètres ont aussi été choisis pour être capables de modéliser le comportement de cibles aux mouvements erratiques, comme les atomes dans des simulations moléculaires interactives --- l’application à l’origine de la problématique abordée dans cette thèse. Nous proposons également une extension de ce modèle, mieux adaptée à certains objets macroscopiques.

	À partir du modèle VFA, nous avons mené une expérience pour connaître l'influence de ces paramètres sur les performances de sélection. Les résultats quantitatifs de cette expérience ont montré qu'il était difficile d'établir un modèle prédictif simple pour déterminer un indice de difficulté de la tâche de sélection, du fait d'une forte interaction entre ces paramètres. Des résultats qualitatifs nous montrent que ce sont les combinaisons particulières de facteurs qui évoquent chez les sujets trois classes de mouvements, perçus comme \og vibratoires \fg{} ou \og browniens \fg{}, \og réguliers \fg{}, ou \og imprévisibles \fg{}. Ces catégories sont directement reliées à la difficulté de sélection et suscitent différentes stratégies d'anticipation. Ces résultats ont conduit à rechercher d'autres critères, notamment des descripteurs de trajectoires de cibles, qui dépendent des facteurs précédemment décrits, comme le périmètre et l'aire de l'enveloppe convexe de la trajectoire que parcourt la cible sur une période donnée. Ces critères permettent de mieux prédire les performances de sélection d'une cible mobile, grâce à différents modèles que nous proposons et détaillons.

	Nous observons que la distance du modèle de Fitts n'a presque pas d'influence sur les performances de sélection de cibles mobiles, si elles sont vives et imprévisibles, puis validons notre prédiction des performances de sélection en montrant qu'il est possible de les améliorer en ajustant les tailles des cibles selon la difficuté prédite. Nous montrons par ailleurs que cette estimation de la difficulté peut être utilisée pour biaiser l'heuristique d'une technique de prédiction intentionnelle afin d'en améliorer les performances.
	
	Nous étudions également l'intérêt d'une assistance pseudo-haptique ajoutée à une technique de prédiction intentionnelle, et montrons qu'elle peut améliorer ou détériorer les performances de sélection, selon la stratégie adoptée par l'utilisateur vis-à-vis du compromis vitesse-précision inhérent à toute tâche de sélection.
	
	Ces résultats obtenus conduisent également à la proposition de recommandations ergonomiques pour la conception de nouvelles techniques de sélection plus adaptées aux cibles en mouvement. Nous nous intéressons tout particulièrement aux différentes combinaisons possibles entre les nombreux principes qui sous-tendent les diverses techniques d'assistance à la sélection, existantes ou introduites dans nos travaux.
 
\textbf{Mots-clefs} : Pointage, \emph{picking} de cibles, sélection de cibles, cibles mobiles, indice de difficulté, aide à la sélection