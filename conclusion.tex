%!TEX root = these.tex

%\chapterstar{CONCLUSION GÉNÉRALE ET PERSPECTIVES}


\chapter*{Conclusion et perspectives}
\addcontentsline{toc}{chapter}{Conclusion}
\mtcaddchapter % pour éviter décalage de minitoc
%\markboth{Conclusion générale et perspectives}{Conclusion générale et perspectives}

	Nous résumons ici les contributions de nos travaux sur la sélection de cibles mobiles, qui sont de natures diverses : états de l'art et classifications, études empiriques, efforts d'interprétation et de modélisation, et recommandations pour la conception de techniques d'assistance à la sélection.

	\section*{Contributions}
	Nous avons dans un premier temps dressé un inventaire assez complet, sans qu'il soit exhaustif, des domaines et des applications impliquant une tâche de sélection de cible mobile, qui est le résultat d'un travail systématique d'entretiens et d'échanges avec les spécialistes de chaque domaine, pour cibler leurs besoins et leur problématique. Ces applications sont très diverses, et nous avons vu que les besoins précis peuvent varier de façon importante d'un contexte à l'autre. Parfois, le temps de sélection prime ; parfois, c'est le taux d'erreurs ; dans certains cas, l'un, l'autre ou les deux peuvent être critiques. Ce travail a mis en lumière les caractéristiques particulières des simulations de dynamique moléculaire (cibles très nombreuses, environnements denses, mouvements rapides et imprévisibles, etc.) qui font de cette application un cas particulièrement difficile, justifiant une attention particulière. Nous en retenons qu'elle peut donc faire office de référence, en partant du principe qu'une technique de sélection efficace pour cette application le serait dans toutes les autres, si ce n'est qu'elle n'impose pas de borne stricte sur le temps de sélection ou le taux d'erreurs.
	
	Nous avons fait un état de l'art des efforts de modélisation de la tâche de sélection de cible, en notant que les travaux portant sur les cibles mobiles étaient rares, et limités aux cibles de mouvement rectiligne, et souvent de vitesse constante. En dressant un état de l'art des techniques d'assistance à la sélection, nous avons observé qu'elles étaient nombreuses et diverses, mais fondées sur les recommandations issues de la loi de Fitts, formulée pour des cibles statiques. De fait, leur efficacité pour des cibles mobiles laisse parfois à désirer, surtout dans les environnements denses, même si elles apportent généralement quelque bénéfice. Certaines techniques cherchent à prédire l'intention de l'utilisateur, et peuvent obtenir de meilleurs résultats, surtout si elles procèdent à partir de l'hypothèse de cibles mobiles, ce qui est parfois le cas. Nous avons noté le potentiel de l'estimation du regard de l'utilisateur (par suivi de la tête, voire des yeux) notamment pour les techniques de prédiction de l'intention, et du filtrage sémantique, notamment par commande vocale. Soulignons également les apports potentiels d'une assistance haptique ou pseudo-haptique, pouvant tout à fait être combinée avec d'autres approches, y compris les plus efficaces.
	
	Nous avons établi une liste de critères permettant de classifier les cibles en fonction de la nature de leurs mouvements (autocorrélés, rapides, prévisibles ou non, etc.) et de leur contexte environnemental (selon la densité et le niveau d'occultation). Puis, nous avons appliqué ces critères afin d'obtenir une classification fine et détaillée des environnements de sélection présentés au cours du premier chapitre. Ensuite, nous avons proposé un modèle opérationnel dans un contexte expérimental --- baptisé \emph{VFA} --- permettant de décrire et générer une large gamme de mouvements, du mouvement brownien au mouvement rectiligne uniforme. Ce modèle est fondé sur trois paramètres décrivant le mouvement d'un objet : sa vitesse, la fréquence de ses changements de direction, et l'amplitude maximale des changements en question. Cela nous a permis d'analyser des cas concrets d'environnements de sélection à travers le prisme de ce modèle, afin d'étayer notre classification des environnements de sélection avec des mesures quantitatives et objectives. Cela nous permet notamment d'identifier les cas présentant les difficultés de sélection les plus importantes, et de guider la conception de techniques d'assistance à la sélection dédiées à ces applications.
	
	Nous avons montré que notre modèle VFA était extensible à la 3D, et pouvait également être étendu pour décrire les mouvements autocorrélés, et surtout pour en générer. Nous avons présenté le protocole et les résultats d'une évaluation empirique des performances de sélection de cibles mobiles en fonction de la nature de leurs mouvements, et avons détaillé l'impact des paramètres du modèle VFA sur les performances. Nous avons montré qu'elles dépendaient fortement de la capacité d'un utilisateur à prédire les mouvements d'une cible, qui peuvent être perçus comme stables ou vibratoires, donc prévisibles et faciles à saisir, ou saccadés, donc imprévisibles et difficiles à saisir. Nous avons proposé le produit AF comme prédicteur de la difficulté de sélection. Nous avons analysé les profils de vitesse du curseur lors des tâches de sélection, observé que la phase de correction est d'autant plus dominante par rapport à la phase balistique que le mouvement est imprévisible, et en avons déduit qu'une technique d'assistance à la sélection devrait se concentrer sur la facilitation de cette phase. Nous avons découvert qu'il était possible et simple, pour une vitesse donnée, d'estimer le périmètre de l'enveloppe convexe($\mathcal{P}(ec)$) d'une trajectoire générée par le modèle VFA à partir de ses paramètres, ainsi que son aire ($\mathcal{A}(ec)$). Nous avons montré que ce périmètre et cette aire étaient des indices fiables pour prédire les performances de sélection. Nous avons également proposé des modèles permettant d'estimer les performances de sélection à partir de $\mathcal{P}(ec)$ ou de $\mathcal{A}(ec)$, donc à partir de V, F et A.
	
	Ainsi, pour une tâche de sélection de cibles mobiles données, nos travaux et les outils que nous avons développés permettent d'extraire les paramètres (V,F,A) de la tâche (par annotation de vidéo ou par estimation), puis de prédire la difficulté de la tâche.
	
	%Parler des résultats opérationnels : on peut extraire les paramètres de la cible, on peut calculer une coque convexe et donc avoir la difficulté et donc de manière générale savoir si une tache décrite au premier chapitre est difficile ou non.
	
	
	\section*{Perspectives}
	La poursuite des travaux sur ces problématiques peut s'articuler autour de plusieurs axes.
	
	Sur un axe théorique, d'abord, pouvoir déterminer analytiquement les paramètres de nos modèles prédisant le temps de sélection en fonction de $\mathcal{P}(ec)$ ou $\mathcal{A}(ec)$ serait fort utile. L'on pourrait également s'attacher à produire des preuves formelles de certains points que nous avançons sur la base de très fortes corrélations, par exemple. Notre modèle VFA est simple, et par conséquent admet quelques limitations. Nous avons proposé des extensions qu'il conviendrait d'explorer, et qui nécessiteraient sans doute de similaires efforts de caractérisation. La modélisation des mouvements autocorrélés, en particulier, soulève de nombreuses questions à explorer, de même que les variations de la vitesse. Nous n'avons pas examiné les éventuelles interactions entre les paramètres définissant la forme d'une cible et ceux de notre modèle VFA, en nous contentant de cibles sphériques. Nous ignorons donc si elles existent, et \emph{a fortiori} de quelle nature elles pourraient être. Plus fondamentalement, si nous avons pu observer des liens entre certaines familles de mouvement et l'impression de prévisibilité ou d'imprévisibilité, les mécanismes cognitifs précis qui régissent l'impression subjective d'un utilisateur nous demeurent inconnus, et devraient faire l'objet de nouveaux travaux --- sans doute de psychologie expérimentale --- afin de mieux comprendre les enjeux perceptifs liés à la nature du mouvement des cibles.
	
	Sur un axe empirique, l'évaluation de notre modèle avec plus de sujets, dans plus de conditions et dans des contextes différents permettrait de confirmer nos résultats sur le pouvoir prédictif du périmètre et de l'aire de l'enveloppe convexe. Citons par exemple leur évaluation dans des contextes immersifs tels que des CAVE ou des casques de réalité virtuelle, ou bien avec de très grands dispositifs tels que les murs d'écrans, avec des environnements virtuels de densités diverses, avec plus ou moins d'occultation, avec des périphériques de saisie pourvus de plus de degrés de liberté tels que des bras haptiques, avec ou sans stéréoscopie (adaptative), etc. Naturellement, la pertinence de notre modèle pour des tâches de sélection assistées par diverses techniques (existantes ou futures) reste une question ouverte. La loi de Fitts, notamment grâce à l'extension des concepts d'amplitude et de largeur à ceux d'amplitude effective et de largeur effective, a pu être adaptée pour modéliser la sélection assistée par de nombreuses techniques. La question de la robustesse de notre modèle dans de telles situations reste totalement ouverte.
	
	D'un point de vue plus applicatif et opérationnel, l'exploitation de nos recommandations pour la conception de techniques de sélection permet d'envisager bien des possibilités : il peut s'agir de suggestions concrètes et combinables à l'envi --- utilisation de l'estimation du regard, d'assistance haptique ou pseudo-haptique, de filtrage sémantique (notamment à reconnaissance vocale) --- mais aussi de recommandations plus ouvertes, concernant la phase de correction au cours du mouvement de sélection, ou encore l'ajustement des paramètres A et F --- sans compter V, dont l'importance n'est pas une surprise. Les possibilités sont d'autant plus nombreuses que certains dispositifs d'affichage et d'interaction offriront des options que d'autres ne permettront pas.
	
	Si nous espérons avoir quelque peu égratigné la surface de la problématique de sélection des cibles mobiles, les travaux présentés ici nous ont continuellement démontré à quel point elle était profonde, complexe et obscure. Puissent les chercheurs qui s'intéresseront au domaine y jeter quelque lumière.