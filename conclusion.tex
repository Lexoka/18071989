%!TEX root = these.tex

%\chapterstar{CONCLUSION GÉNÉRALE ET PERSPECTIVES}


\chapter*{Conclusion et perspectives}
\addcontentsline{toc}{chapter}{Conclusion}
\mtcaddchapter % pour éviter décalage de minitoc
%\markboth{Conclusion générale et perspectives}{Conclusion générale et perspectives}

	Nous résumons ici les contributions de nos travaux sur la sélection de cibles mobiles, qui sont de natures diverses : états de l'art et classifications, études empiriques, efforts d'interprétation et de modélisation, propositions et évaluations de nouvelles techniques, et recommandations pour leur amélioration ou la création de nouvelles techniques supplémentaires.

	\section*{Contributions}
	Nous avons dans un premier temps dressé un inventaire assez riche des domaines et des applications impliquant une tâche de sélection de cible mobile, qui résulte d'un travail systématique d'entretiens et d'échanges avec les spécialistes de chaque domaine, pour cibler leurs besoins et leurs problématiques. Ces applications sont très diverses, et nous avons vu que les besoins précis peuvent varier de façon importante d'un contexte à l'autre. Parfois, le temps de sélection prime ; parfois, c'est le taux d'erreurs ; dans certains cas, l'un, l'autre ou les deux peuvent être critiques. Ce travail a mis en lumière les caractéristiques particulières des simulations de dynamique moléculaire (cibles très nombreuses, environnements denses, mouvements rapides et imprévisibles, etc.) qui font de cette application un cas particulièrement difficile, justifiant une attention particulière. Nous n'en oublions pas pour autant les spécificités et contraintes des autres applications.
	
	Nous avons fait un état de l'art des efforts de modélisation de la tâche de sélection de cible, en notant que les travaux portant sur les cibles mobiles étaient rares, et limités aux cibles de mouvement rectiligne, et souvent de vitesse constante. Via un état de l'art des techniques d'assistance à la sélection, nous avons observé qu'elles étaient nombreuses et diverses, mais fondées sur les recommandations issues de la loi de Fitts, formulée pour des cibles statiques. De fait, elles sont souvent peu adaptées aux cibles mobiles, surtout dans les environnements les plus denses, même si elles apportent généralement quelque bénéfice. Certaines techniques cherchent à prédire l'intention de l'utilisateur, et peuvent obtenir de meilleurs résultats, surtout si elles procèdent à partir de l'hypothèse de cibles mobiles, ce qui est parfois le cas. Nous avons noté le potentiel de l'estimation du regard de l'utilisateur (par suivi de la tête, voire des yeux) notamment pour les techniques de prédiction de l'intention, et du filtrage sémantique, par exemple par commande vocale. Nou soulignons également les apports potentiels d'une assistance haptique ou pseudo-haptique, pouvant tout à fait être combinée avec d'autres approches, y compris les plus efficaces.
	
	Nous avons établi une liste de critères permettant de classifier les types de mouvement rencontrés pour les diverses cibles mobiles examinées : autocorrélés, rapides, prévisibles ou non, etc. Nos critères tiennent également compte de leur contexte environnemental (selon la densité et le niveau d'occultation). Nous avons appliqué ces critères afin d'obtenir une classification fine et détaillée des environnements de sélection présentés au cours du premier chapitre. Ensuite, nous avons proposé un modèle opérationnel dans un contexte expérimental --- baptisé \emph{VFA} --- permettant de décrire et générer une vaste gamme de mouvements, du mouvement brownien au mouvement rectiligne uniforme. Ce modèle est fondé sur trois paramètres décrivant le mouvement d'un objet : sa vitesse, la fréquence de ses changements de direction, et l'amplitude maximale des changements en question. Cela nous a permis d'analyser des cas concrets d'environnements de sélection à travers le prisme de ce modèle, afin d'étayer notre classification des environnements de sélection avec des mesures quantitatives et objectives. Ainsi, nous avons pu notamment identifier les cas présentant les difficultés de sélection les plus importantes, afin de guider la conception de techniques d'assistance à la sélection dédiées à ces applications. Nous avons proposé deux extensions du modèle VFA pouvant décrire les mouvements autocorrélés (c'est-à-dire les mouvements dont les changements de direction à un instant influencent ceux de l'instant suivant), et surtout pouvant en générer. Nous avons aussi montré que notre modèle VFA était aisément extensible à la 3D.
	
	Nous avons présenté le protocole et les résultats d'une évaluation empirique des performances de sélection de cibles mobiles en fonction de la nature de leurs mouvements, et avons détaillé l'effet des paramètres du modèle VFA sur les performances. Nous avons montré qu'elles dépendaient fortement de la capacité d'un utilisateur à prédire les mouvements d'une cible, qui peuvent être perçus comme stables ou vibratoires, donc prévisibles et faciles à saisir, ou saccadés, donc imprévisibles et difficiles à saisir. Nous avons proposé le produit AF comme prédicteur de la difficulté de sélection. Nous avons analysé les profils de vitesse du curseur lors des tâches de sélection, observé que les phases lentes sont d'autant plus dominantes par rapport aux phases rapides que le mouvement est imprévisible, et en avons déduit qu'une technique d'assistance à la sélection devrait se concentrer sur la facilitation des phases lentes. Nous avons découvert qu'il était possible et simple, pour une vitesse donnée, d'estimer l'espérance du périmètre de l'enveloppe convexe($\mathbb{E}(\mathcal{P}(ec))$) d'une trajectoire générée par le modèle VFA à partir de ses paramètres, ainsi que l'espérance de son aire ($\mathbb{E}(\mathcal{A}(ec))$). Nous avons montré que ces espérances de périmètre et d'aire étaient des indices fiables pour prédire les performances de sélection. Nous avons également proposé des modèles permettant d'estimer les performances de sélection à partir de $\mathbb{E}(\mathcal{P}(ec))$ ou de $\mathbb{E}(\mathcal{A}(ec))$, donc à partir de V, F et A.
	
	Ainsi, pour une tâche de sélection de cibles mobiles données, nos travaux et les outils que nous avons développés permettent d'extraire les paramètres (V,F,A) de la tâche (par annotation de vidéo ou par estimation), puis de prédire la difficulté de la tâche.
	
	Nous avons proposé plusieurs méthodes de modification du comportement d'une cible, par filtrage ou lissage, informées par notre modèle VFA et l'estimation de la difficulté de sélection qu'il permet. Nous avons montré par une petite étude empirique que la distance de Fitts n'était plus un facteur majeur de difficulté dans la sélection de cibles mobiles rapides et vives. Puis, une autre étude empirique nous a permis de valider notre modèle d'estimation de la difficulté de sélection, notamment en montrant qu'il était possible d'améliorer les performances globales de sélection en ajustant les tailles des cibles selon leurs difficultés respectives, et ce sans augmenter leur encombrement visuel global. De façon analogue, nous avons biaisé l'heuristique de prédiction d'une technique existante, \emph{Hook}, pour en créer une variante mieux à même de sélectionner les cibles difficiles, sans trop nuire à la sélection des cibles faciles, de sorte que les performances globales s'en trouvent améliorées, selon une étude que nous avons menée. Puis, nous avons évalué l'apport d'une assistance pseudo-haptique à une telle technique, dans un contexte de difficulté extrême, et montré qu'il dépendait de la stratégie des utilisateurs. Nous avons montré que dans un tel contexte, l'utilisation combinée d'élage des distracteurs et de prédiction intentionnelle permet de bien meilleures performances de sélection. Enfin, nous avons critiqué l'ensemble de ces résultats, et discuté de potentielles voies d'amélioration de ces techniques.
	
	\section*{Perspectives}
	La poursuite des travaux sur ces problématiques peut s'articuler autour de plusieurs axes.
	
	Sur un axe théorique, d'abord, pouvoir déterminer analytiquement les relations exactes entre V, F, A, $\mathbb{E}(\mathcal{P}(ec))$, $\mathbb{E}(\mathcal{A}(ec))$ et les performances de sélection serait fort utile. L'on pourrait également s'attacher à produire des preuves formelles de certains points que nous avançons sur la base de très fortes corrélations, par exemple. Notre modèle VFA est simple, et par conséquent admet quelques limitations. Nous avons proposé des extensions qu'il conviendrait d'explorer, et qui nécessiteraient sans doute de similaires efforts de caractérisation. La modélisation des mouvements autocorrélés, en particulier, soulève de nombreuses questions à explorer, de même que les variations de la vitesse. Nous n'avons pas examiné les éventuelles interactions entre les paramètres définissant la forme d'une cible et ceux de notre modèle VFA, en nous contentant de cibles sphériques. Nous ignorons donc si elles existent, et \emph{a fortiori} de quelle nature elles pourraient être. Plus fondamentalement, si nous avons pu observer des liens entre certaines familles de mouvement et l'impression de prévisibilité ou d'imprévisibilité, les mécanismes cognitifs précis qui régissent l'impression subjective d'un utilisateur nous demeurent inconnus, et devraient faire l'objet de nouveaux travaux --- sans doute de psychologie expérimentale --- afin de mieux comprendre les enjeux perceptifs liés à la nature du mouvement des cibles.
	
	Sur un axe empirique, l'évaluation de notre modèle avec plus de sujets, dans plus de conditions et dans des contextes différents permettrait de confirmer nos résultats sur le pouvoir prédictif du périmètre et de l'aire de l'enveloppe convexe. Citons par exemple leur évaluation dans des contextes immersifs tels que des CAVE ou des casques de réalité virtuelle, ou bien avec de très grands dispositifs tels que les murs d'écrans, avec des environnements virtuels de densités diverses, avec plus ou moins d'occultation, avec des périphériques de saisie pourvus de plus de degrés de liberté tels que des bras haptiques, avec ou sans stéréoscopie (adaptative), etc. De même, il nous apparaît que si notre modèle prédit avec une certaine fiabilité la difficulté de sélection d'une cible avec un pointeur ponctuel simple, il n'est pas nécessairement optimal lorsqu'une technique d'assistance à la sélection est utilisée. Sans doute serait-il utile de développer des modèles similaires pour chaque technique, sans parler de comprendre les lois fondamentales qui déterminent les paramètres d'un modèle en fonction de la technique employée. La loi de Fitts, notamment grâce à l'extension des concepts de distance et de largeur à ceux de distance effective et de largeur effective, a pu être adaptée pour modéliser la sélection assistée par de nombreuses techniques. La question de la robustesse de notre modèle dans de telles situations reste relativement ouverte, bien que les résultats observés avec \emph{Hook}, \emph{SharpHook} et \emph{FastHook} soient relativement encourageants à cet égard.
	
	D'un point de vue plus applicatif et opérationnel, l'exploitation de nos recommandations pour la conception de techniques de sélection permet d'envisager bien des possibilités : le biais d'une heuristique de prédiction comme nous l'avons évalué, ou encore l'assistance pseudo-haptique, mais aussi l'assitance haptique, l'utilisation de l'estimation du regard, les lissages et filtrages de trajectoires, le filtrage sémantique (notamment à reconnaissance vocale)\ldots{} Les possibilités sont d'autant plus nombreuses que certains dispositifs d'affichage et d'interaction offrent des options spécifiques. Dans certains cas, par exemple, il pourait être souhaitable de manipuler deux curseurs à la fois, un avec chaque main, particulièrement pour nourrir une heuristique de prédiction intentionelle.
	
	Si nous espérons avoir quelque peu égratigné la surface de la problématique de sélection des cibles mobiles, les travaux présentés ici nous ont continuellement démontré à quel point elle était profonde, complexe et obscure. Puissent les chercheurs qui s'intéresseront au domaine y jeter quelque lumière.