%!TEX root = these.tex

%\chapterstar{CONCLUSION GÉNÉRALE ET PERSPECTIVES}


\chapter*{Conclusion et perspectives}
\addcontentsline{toc}{chapter}{Conclusion}

%\mtcaddchapter
%\markboth{Conclusion générale et perspectives}{Conclusion générale et perspectives}

	Nous avons dans un premier temps dressé un inventaire des domaines et des applications impliquant une tâche de sélection de cible mobile. Ceux-ci sont très divers, et nous avons vu que les besoins précis peuvent varier de façon importante d'un contexte à l'autre. Parfois, le temps de sélection prime ; parfois, c'est le taux d'erreurs ; dans certains cas, l'un, l'autre ou les deux peuvent être critiques. Ce travail a mis en lumière les caractéristiques particulières des simulations de dynamique moléculaire (cibles très nombreuses, environnements denses, mouvements rapides et imprévisibles, etc.) qui font de cette application un cas particulièrement difficile, justifiant une attention particulière. Nous en retenons qu'elle peut donc faire office de référence, en partant du principe qu'une technique de sélection efficace pour cette application le serait dans toutes les autres, si ce n'est qu'elle n'impose pas de borne stricte sur le temps de sélection ou le taux d'erreurs.
	
	Nous avons fait un état de l'art des efforts de modélisation de la tâche de sélection de cible, en notant que les travaux portant sur les cibles mobiles étaient rares, généralement limités aux cibles de mouvement rectiligne, et souvent de vitesse constante. En dressant un état de l'art des techniques d'assistance à la sélection, nous avons observé qu'elles étaient nombreuses et diverses, mais généralement fondés sur les recommandations issues de la loi de Fitts, formulée pour des cibles statiques. De fait, leur efficacité pour des cibles mobiles laisse souvent à désirer, même si elles apportent généralement quelque bénéfice. Certaines techniques cherchent à prédire l'intention de l'utilisateur, et peuvent obtenir de meilleurs résultats, surtout si elles procèdent à partir de l'hypothèse de cibles mobiles, ce qui est parfois le cas. Nous avons noté le potentiel de l'estimation du regard de l'utilisateur (par suivi de la tête, voire des yeux) notamment pour les techniques de prédiction de l'intention, et du filtrage sémantique, notamment par commande vocale. Soulignons également les apports potentiels d'une assistance haptique ou pseudo-haptique, pouvant tout à fait être combinée avec d'autres approches, y compris les plus efficaces.
	
	