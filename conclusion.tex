%!TEX root = these.tex

%\chapterstar{CONCLUSION GÉNÉRALE ET PERSPECTIVES}


\chapter*{Conclusion et perspectives}
\addcontentsline{toc}{chapter}{Conclusion}
\mtcaddchapter % pour éviter décalage de minitoc
%\markboth{Conclusion générale et perspectives}{Conclusion générale et perspectives}

	Nous résumons ici les contributions de nos travaux sur la sélection de cibles mobiles, qui sont de natures diverses : états de l'art et classifications, études empiriques, efforts d'interprétation et de modélisation, et recommandations pour la conception de techniques d'assistance à la sélection.

	\section{Contributions}
	Nous avons dans un premier temps dressé un inventaire des domaines et des applications impliquant une tâche de sélection de cible mobile. Ceux-ci sont très divers, et nous avons vu que les besoins précis peuvent varier de façon importante d'un contexte à l'autre. Parfois, le temps de sélection prime ; parfois, c'est le taux d'erreurs ; dans certains cas, l'un, l'autre ou les deux peuvent être critiques. Ce travail a mis en lumière les caractéristiques particulières des simulations de dynamique moléculaire (cibles très nombreuses, environnements denses, mouvements rapides et imprévisibles, etc.) qui font de cette application un cas particulièrement difficile, justifiant une attention particulière. Nous en retenons qu'elle peut donc faire office de référence, en partant du principe qu'une technique de sélection efficace pour cette application le serait dans toutes les autres, si ce n'est qu'elle n'impose pas de borne stricte sur le temps de sélection ou le taux d'erreurs.
	
	Nous avons fait un état de l'art des efforts de modélisation de la tâche de sélection de cible, en notant que les travaux portant sur les cibles mobiles étaient rares, généralement limités aux cibles de mouvement rectiligne, et souvent de vitesse constante. En dressant un état de l'art des techniques d'assistance à la sélection, nous avons observé qu'elles étaient nombreuses et diverses, mais généralement fondés sur les recommandations issues de la loi de Fitts, formulée pour des cibles statiques. De fait, leur efficacité pour des cibles mobiles laisse souvent à désirer, même si elles apportent généralement quelque bénéfice. Certaines techniques cherchent à prédire l'intention de l'utilisateur, et peuvent obtenir de meilleurs résultats, surtout si elles procèdent à partir de l'hypothèse de cibles mobiles, ce qui est parfois le cas. Nous avons noté le potentiel de l'estimation du regard de l'utilisateur (par suivi de la tête, voire des yeux) notamment pour les techniques de prédiction de l'intention, et du filtrage sémantique, notamment par commande vocale. Soulignons également les apports potentiels d'une assistance haptique ou pseudo-haptique, pouvant tout à fait être combinée avec d'autres approches, y compris les plus efficaces.
	
	Nous avons établi une liste de critères permettant de classifier les cibles en fonction de la nature de leurs mouvements (autocorrélés, rapides, prévisibles ou non, etc.) et de leur contexte environnemental (selon sa densité et son niveau d'occultation). Puis, nous avons appliqué ces critères afin d'obtenir une classification fine et détaillée des environnements de sélection présentés au cours du premier chapitre. Ensuite, nous avons proposé un modèle --- baptisé \emph{VFA} --- permettant de décrire (et générer) du mouvement aléatoire. Celui-ci est fondé sur trois paramètres décrivant le mouvement d'un objet : sa vitesse, la fréquence de ses changements de direction, et l'amplitude maximale des changements en question. Cela nous a permis d'analyser des cas concrets d'environnements de sélection à travers le prisme de ce modèle, afin d'étayer notre classification des environnements de sélection avec des mesures quantitatives et objectives. Cela nous permet notamment d'identifier les cas présentant les difficultés de sélection les plus importantes, et de guider la conception de techniques d'assistance à la sélection dédiées à ces applications.
	
	Nous avons montré que notre modèle VFA était extensible à la 3D, et pouvait également être étendu pour décrire les mouvements autocorrélés, et surtout pour en générer. Nous avons présenté le protocole et les résultats d'une évaluation empirique des performances de sélection de cibles mobiles en fonction de la nature de leurs mouvements, et avons détaillé l'impact des paramètres du modèle VFA sur les performances. Nous avons montré qu'elles dépendaient fortement de la capacité d'un utilisateur à prédire les mouvements d'une cible, qui peuvent être perçus comme stables ou vibratoires, donc prévisibles et faciles à saisir, ou saccadés, donc imprévisibles et difficiles à saisir. Nous avons proposé le produit FA comme prédicteur de la difficulté de sélection. Nous avons analysé les profils de vitesse du curseur lors des tâches de sélection, observé que la phase de correction est d'autant plus dominante par rapport à la phase balistique que le mouvement est imprévisible, et en avons déduit qu'une technique d'assistance à la sélection devrait se concentrer sur la facilitation de cette phase. Nous avons découvert qu'il était possible et simple, pour une vitesse donnée, d'estimer l'aire de l'enveloppe convexe $\mathcal{A}(ec)$ d'une trajectoire générée par le modèle VFA à partir de ses paramètres, et que cette aire pouvait prédire les performances de sélection avec une certaine efficacité. Nous avons également proposé un modèle permettant d'estimer les performances de sélection à partir de $\mathcal{A}(ec)$, donc à partir de V, F et A.
	
	\section{Perspectives}
	Plein !