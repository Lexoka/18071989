%!TEX root = these.tex

\chapter[Contexte/Besoin/Applications]{plif}
\minitoc
\label{chap1}
\cleardoublepage

    \section{Contrôle du trafic aérien}
    La figure~\ref{fig:airtraffic} représente un écran de contrôle du trafic aérien. Les trajectoires des avions sont (normalement) légèrement courbées ou rectilignes, ce qui les rend particulièrement prévisibles, et donc facilite considérablement la tâche de sélection, comme nous le verrons en détail plus loin. Cependant, selon le niveau de zoom et la quantité d'informations contextuelles affichées sur l'écran, le niveau d'occultation peut devenir très important. La vitesse des cibles dépend également du niveau de zoom, mais dans une relation inverse : plus l'échelle est grande, plus les mouvements des avions seront lents sur l'écran.
    
    Bien sûr, cette tâche a aussi la particularité d'être critique, puisque des centaines de vies sont en jeu. C'est un élément dont il faut tenir compte si l'on développe une technique d'interaction pour le contrôle aérien. En effet, au-delà du temps moyen de sélection d'une cible, le temps maximal acceptable et le taux d'erreur sont cruciaux.
    
    Les systèmes de contrôle aérien dont nous avons connaissance sont tous en deux dimensions. Une technique de sélection suffisamment performante pourrait rendre envisageable l'utilisation d'un système en 3D. Cela permettrait par exemple de mieux déterminer si deux avions qui paraissent dangereusement proches dans le plan le sont réellement dans l'espace, ou tout simplement d'avoir une meilleure représentation générale de la situation.
    
	\begin{figure}[h]
		\centering
		\includegraphics[width=\textwidth]{figures/Radar-Scope-ZSE}
		\caption{Écran de contrôle du trafic aérien. Crédit : www.boldmethod.com}
		\label{fig:airtraffic}
	\end{figure}
	
	\section{Vidéo-surveillance}
	La vidéo-surveillance s'applique aussi bien aux foules dans les lieux publics ou sensibles qu'à la voirie, où peuvent circuler des véhicules de types divers. Un agent de sécurité en charge de visionner un flux de vidéo-surveillance pourrait avoir à sélectionner une personne, par exemple pour obtenir des informations sur celle-ci grâce à la reconnaissance faciale, ou un véhicule, pour les mêmes raisons, par exemple grâce à sa plaque d'immatriculation. Quelle que soit la nature de la cible, sa sélection pourrait avoir pour but de zoomer dessus, de verrouiller une caméra robotisée afin qu'elle la suive, ou de la désigner à des forces de sécurité sur le terrain pour qu'elles interviennent physiquement.
	
	\section{Analyse de processus complexes}
	
	\section{Retransmissions d'événements sportifs}
	Les retransmissions d'événements sportifs présentent un potentiel d'interactivité intéressant, nécessitant souvent la sélection de cibles mobiles. Il peut être intéressant, par exemple, de sélectionner un joueur de football pour afficher des statistiques. Pour les mêmes raisons, un ballon ou une balle peut être une cible, de même que certains éléments du terrain de jeu, qui sont physiquement fixes mais peuvent être mobiles à l'écran du fait des mouvements de caméra. La nature du mouvement de ces cibles varie nécessairement d'un sport ou d'un jeu à l'autre.
	
	Des sports d'équipe comme le football, le hockey sur glace ou le basket-ball sont caractérisés par des cibles potentielles relativement nombreuses, mobiles, et dont les mouvements ne sont pas toujours très prévisibles. Les courses athléthiques, hippiques, ou les sports mécaniques présentent des mouvement souvent plus prévisibles, mais aussi nettement plus rapides, avec en plus une certaine tendance des cibles à être très proches les unes des autres, ce qui augmente la probabilité d'erreur de sélection.
	
	Cependant, dans la plupart 
	
	\section{Jeux vidéo}
	La sélection ou le pointage de cibles mobiles est une tâche que l'on retrouve dans de très nombreux jeux vidéo, appartenant à un nombre important de catégories. On peut notamment citer les jeux de tir (\emph{Fist-person Shooters}, ou FPS), les jeux de stratégie et tactique militaire (les catégories \emph{Real-time strategy} ou RTS, et \emph{Real-time tactics} ou RTT), les simulateurs de vol de combat (réalistes ou non) ainsi que les jeux de type arène de bataille en ligne multijoueur (\emph{Multiplayer online battle arena}, ou MOBA).
	
	
	
	\section{}
    
    Les tendances observées sur les cibles statiques demeurent, à savoir que la difficulté de la tâche augmente quand la distance entre la cible et le pointeur augmente, ou quand la taille de la cible diminue. Mais l'influence du mouvement de la cible empêche d'appliquer la loi de Fitts. Certes, la difficulté de la tâche augmente aussi avec la vitesse, [ref-mec-interact?] cependant, si la nature du mouvement a une importance cruciale, celle-ci demeure peu étudiée.
    
    
    
    
 The
difficulty of the selection increases while (1) target size decreases,
(2) target velocity increases and (3) target density increases. The
selection is even more difficult in 3D environments (filmed or synthetic
environments) because a target of interest can be occluded
by others targets. Selecting someone in crowds with a surveillance
video system is an example of a dense environment with small targets
and occlusion. In some cases, like in action sport footage, both
the objects of interest and the camera can move. Target movements
become unpredictable, and the selection is even more difficult.
In this context, as Hasan et al. [3] wrote, for completing the
selection ”the user must continually track the target and simultaneously
plan to move the cursor over it”. This underscores the key
point of the technique we propose here. Since the user follows the
target for selecting it, the Hook technique tracks the cursor behavior
for assisting the selection. Indeed, observing the history of cursor
displacements, and the history of distances between each target and
the cursor, the system can estimate which target is tracked, and then
propose a selection to the user who just has to validate it. In other
words, the user follows the target of interest, and the system will
know which target it is.
This paper presents the implementation of this technique, and
two experiments which investigate its performance. Hook has been
compared to the basic pointing (non assisted pointing), and to Bubble
cursor [2, 8], for the case of 3D object selection. The first evaluation
involves a desktop configuration, in which pointing is done
on a standard screen with a mouse. The second evaluation involves
an immersive configuration in which pointing is done with a 3dof
(degrees-of-freedom) device. Both evaluations highlight the bene-
fits of this new interaction technique for selecting moving targets in
a 3D environment. More than simply improving the pointing time,
Hook drastically decreases the error rate and allows pointing targets
in high density environments, with high velocity targets that are not
possible to capture with other techniques.


	
		

\clearpage
