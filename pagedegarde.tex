\begin{titlepage}

\includegraphics[height=2.cm]{./figures/garde/logo_paris_saclay}\hfill
\includegraphics[height=2.cm]{./figures/garde/logo_CNRS}\hfill
\includegraphics[height=2.cm]{./figures/garde/limsilogo_new_transparent_crop}\hfill
\\
\\


\begin{center}
 \textbf{THÈSE DE DOCTORAT DE\\ L'UNIVERSITÉ PARIS SACLAY\\}
\vspace{\stretch{0.5}}
Spécialité\,:\\
\textbf{Informatique}\\ 
\vspace{\stretch{0.5}}
Présentée par\,:\\ 
\vspace{\stretch{0.5}}
\begin{LARGE}
Alexandre KOUYOUMDJIAN\end{LARGE}\\
\vspace{\stretch{1}}
Pour obtenir le grade de\\
\textbf{DOCTEUR DE L'UNIVERSITÉ PARIS SACLAY}
\end{center}

\vspace{\stretch{2}}
\noindent \underline{Sujet de la thèse :}\\
\begin{center}
\begin{Large}
{\textsc{Caractérisation de la sélection de cibles en mouvement : vers la conception de paradigmes d'aide à la sélection de cibles mobiles}}
\end{Large}
\end{center}

\vspace{\stretch{2}}

Soutenue le XX yyyyyyyy 2017, devant le jury composé de :\\

\begin{center}
	\begin{tabular}{l l l}
	
	Thierry Duval		& Professeur d'IMT Atlantique				& Rapporteur	\\ 
	Emmanuel Dubois		& Professeur de l'Université de Toulouse	& Rapporteur	\\
						& 	&				\\ % blank row
	?					&?		 		& Examinateur	\\ 
	?					&?				& Examinateur	\\ 
						& 	&				\\ % blank row
	Nicolas Férey		& Maître de conférences					& Encadrant scientifique\\
						& 	&				\\ % blank row
	Patrick Bourdot 	& Directeur de Recherche, CNRS	& Co-Directeur de thèse\\ 
	Stéphane Huot		& Directeur de Recherche, INRIA	& Co-Directeur de thèse\\ 
		
	\end{tabular}
\end{center}

\vspace{\stretch{2}}

\setlength{\columnsep}{7mm}
\setlength{\columnseprule}{0pt}

\begin{multicols}{2} 
\small 
\noindent Groupe VENISE					\\	
\noindent LIMSI-CNRS					\\
\noindent Rue John von Neumann			\\
\noindent 91403 Orsay Cedex, France		\\	

\columnbreak

\raggedleft Ex Situ										\\
\noindent LRI-INRIA										\\
\noindent Bât. 650 Ada Lovelace, Université Paris-Sud	\\
\noindent 91405 Orsay Cedex, France
\end{multicols}



\end{titlepage}